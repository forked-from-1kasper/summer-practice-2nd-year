\documentclass{article}
\usepackage{amsmath}
\usepackage{amssymb}
\usepackage{amsthm}

\usepackage[utf8]{inputenc}
\usepackage[T2A]{fontenc}
\usepackage[russian]{babel}
\usepackage[tmargin=1in,bmargin=1in,lmargin=1.25in,rmargin=1.25in]{geometry}

\begin{document}

\begin{titlepage}
  \centering
  Федеральное государственное автономное\par
  образовательное учреждение\par
  высшего образования\par
  {\scshape «Сибирский федеральный университет»\par}

  \vspace{0.3cm}
  Институт математики и фундаментальной информатики СФУ\par
  {\small Кафедра математического анализа и дифференциальных уравнений\par}

  \vspace{0.5cm}
  \begin{flushright}
    \begin{tabular}{l@{}}
      {\scshape Утверждаю}\\
      Заведующий кафедрой\\
      $\underset{\text{подпись}}{\underline{\hspace{1.3cm}}}$
      $\underset{\text{фамилия, инициалы}}{\text{Фроленков И. В.}}$\\
      «\underline{\hspace{1.2cm}}» \underline{\hspace{1.2cm}} 20\underline{\hspace{0.4cm}} г.
    \end{tabular}
  \end{flushright}

  \vspace{2cm}
  {\Large\bf Отчёт о практике\par
    по получению профессиональных умений\par
    и опыта профессиональной деятельности\par}

  \vspace{0.5cm}
  Место прохождения практики:\par
  Институт математики и фундаментальной информатики\par

  \vspace{0.5cm}
  Тема практики:\par
  {\bf Последовательность с произвольно заданным\par
       замкнутым множеством предельных точек\par}

  \vspace{1.5cm}
  \begin{tabular}{l}
    Руководитель:
      $\underset{\text{подпись, дата}}{\underline{\hspace{2.2cm}}}$
      $\underset{\text{должность, учёная степень}}{\underline{\hspace{4.6cm}}}$
      $\underset{\text{фамилия, инициалы}}{\underline{\hspace{3.4cm}}}$
    \vspace{0.3cm} \\
    Студент:
      $\underset{\text{номер группы}}{\underline{\hspace{2.2cm}}}$
      $\underset{\text{номер зачётной книжки}}{\underline{\hspace{3.4cm}}}$
      $\underset{\text{подпись, дата}}{\underline{\hspace{2.2cm}}}$
      $\underset{\text{фамилия, инициалы}}{\underline{\hspace{3.4cm}}}$
  \end{tabular}

  \vfill
  Красноярск, 2022

\end{titlepage}

\newcommand{\N}{\mathbb{N}}
\newcommand{\Z}{\mathbb{Z}}
\newcommand{\R}{\mathbb{R}}

\section{Основные определения}

Будем работать в произвольной достаточно общей теории множеств с аксимой выбора.
Определим $\R$ как некоторое непрерывное линейно упорядоченное поле, без доказательств
приводя утверждения о том, что такое поле единственно с точностью до изоморфизма.

блаблабла

\section{Построение искомой последовательности}

блаблабла

\end{document}
