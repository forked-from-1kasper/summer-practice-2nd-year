\documentclass{article}
\usepackage{amsmath}
\usepackage{amssymb}
\usepackage{amsthm}

\usepackage[utf8]{inputenc}
\usepackage[T2A]{fontenc}
\usepackage[russian]{babel}
\usepackage[tmargin=1in,bmargin=1in,lmargin=1.25in,rmargin=1.25in]{geometry}

\newtheorem{theorem}{Теорема}
\newtheorem*{theorem*}{Теорема}
\newtheorem{consequence}{Следствие}
\newtheorem*{consequence*}{Следствие}
\newtheorem{lemma}{Лемма}
\newtheorem*{lemma*}{Лемма}

\begin{document}

\begin{titlepage}
  \centering
  Федеральное государственное автономное\par
  образовательное учреждение\par
  высшего образования\par
  {\scshape «Сибирский федеральный университет»\par}

  \vspace{0.3cm}
  Институт математики и фундаментальной информатики СФУ\par
  {\small Кафедра математического анализа и дифференциальных уравнений\par}

  \vspace{0.5cm}
  \begin{flushright}
    \begin{tabular}{l@{}}
      {\scshape Утверждаю}\\
      Заведующий кафедрой\\
      $\underset{\text{подпись}}{\underline{\hspace{1.3cm}}}$
      $\underset{\text{фамилия, инициалы}}{\text{Фроленков И. В.}}$\\
      «\underline{\hspace{1.2cm}}» \underline{\hspace{1.2cm}} 20\underline{\hspace{0.4cm}} г.
    \end{tabular}
  \end{flushright}

  \vspace{2cm}
  {\Large\bf Отчёт о практике\par
    по получению профессиональных умений\par
    и опыта профессиональной деятельности\par}

  \vspace{0.5cm}
  Место прохождения практики:\par
  Институт математики и фундаментальной информатики\par

  \vspace{0.5cm}
  Тема практики:\par
  {\bf Последовательность с произвольно заданным\par
       замкнутым множеством предельных точек\par}

  \vspace{1.5cm}
  \begin{tabular}{l}
    Руководитель:
      $\underset{\text{подпись, дата}}{\underline{\hspace{2.2cm}}}$
      $\underset{\text{должность, учёная степень}}{\underline{\hspace{4.6cm}}}$
      $\underset{\text{фамилия, инициалы}}{\underline{\hspace{3.4cm}}}$
    \vspace{0.3cm} \\
    Студент:
      $\underset{\text{номер группы}}{\underline{\hspace{2.2cm}}}$
      $\underset{\text{номер зачётной книжки}}{\underline{\hspace{3.4cm}}}$
      $\underset{\text{подпись, дата}}{\underline{\hspace{2.2cm}}}$
      $\underset{\text{фамилия, инициалы}}{\underline{\hspace{3.4cm}}}$
  \end{tabular}

  \vfill
  Красноярск, 2022

\end{titlepage}

\newcommand{\N}{\mathbb{N}}
\newcommand{\Z}{\mathbb{Z}}
\newcommand{\Q}{\mathbb{Q}}
\newcommand{\R}{\mathbb{R}}
\newcommand{\card}{\mathrm{card}}
\newcommand{\Int}{\mathrm{Int}}

\section{Основные определения}

Будем работать в произвольной достаточно общей теории множеств с аксимой выбора.
Определим $\R$ как некоторое непрерывное линейно упорядоченное поле, без доказательств
ссылаясь на утверждения о том, что такое поле единственно с точностью до изоморфизма и существует.

Всюду будем использовать стандартные логические и теоретико-можественные обозначения ($\forall$, $\exists$, $\land$, $\lor$, $\Rightarrow$, $\Leftrightarrow$, $\cup$, $\cap$, $\setminus$, $\varnothing$ и так далее) и,
также стандартно, $f \circ g$ для композиции функций (то есть для функции $x \mapsto f(g(x))$), $f(A)$ для образа (множества $\{ f(x)\ |\ x \in A \}$),
$f^{-1}(B)$ для прообраза (множества $\{ x\ |\ f(x) \in B \}$) и $\card(X)$ для мощности множества.

Множество {\bf счётно,} если его мощность не превышает мощность множества натуральных чисел $\N$.

Точку $x$ множества $A \subseteq \R$ назовём {\bf внутренней,} если $\exists \varepsilon > 0: \forall y: |x - y| < \varepsilon \Rightarrow y \in A$,
то есть если она входит во множество $A$ вместе с некоторой окрестностью.

Множество внутренних точек множества $A$ назовём его {\bf внутренностью} и обозначим как $\Int(A)$.

Множество $A$ назовём {\bf открытым,} если все его точки внутренние или, что то же самое, если оно совпадает со своей внутренностью.
Очевидно, $\Int(A)$ всегда открыто.

Точку $x$ множества $A \subseteq \R$ назовём {\bf предельной,} если в любой окрестности этой точки существует хотя бы одна точка из $A$, не совпадающая с $x$.

Точку множества $A \subseteq \R$ назовём {\bf граничной,} если в любой её окрестности находятся точки, как принадлежащие $A$, так и не принадлежащие $A$.
Множество всех граничных точек называется {\bf границей} и обозначается $\partial A$. Эквивалентно,
\begin{equation*}
  x \in \partial A \Leftrightarrow\ \forall \varepsilon > 0: \exists y \in A, y^\prime \in \R \setminus A: |x - y| < \varepsilon \land |x - y^\prime| < \varepsilon
\end{equation*}

Объединение множества $A$ со множеством всех его предельных точек называется {\bf замыканием} и обозначается $\overline{A}$.

Множество назовём {\bf замкнутым,} если оно содержит все свои предельные точки.

{\bf Числовой последовательностью} назовём всякую функцию $f : \N \rightarrow \R$.

Будем говорить, что число $A \in \R$ является {\bf пределом последовательности} $f$, если выполняется
\begin{equation*}
  \forall \varepsilon > 0: \exists N \in \N: \forall n \geq N : |f(n) - A| \leq \varepsilon
\end{equation*}

В этом случае пишут $\underset{n \rightarrow + \infty}{\lim} f(n) = A$.

Последовательность, у которой существует предел, назовём {\bf сходящейся.}

Если $g : \N \rightarrow \N$ — строго возрастающая последовательность, то есть для неё выполняется
\begin{equation*}
  \forall n, m: n < m \Rightarrow g(n) < g(m),
\end{equation*}

то последовательность $f \circ g$ называется {\bf подпоследовательностью} последовательности $f$.

{\bf Предельной точкой последовательности} называется предел любой её подпоследовательности, если таковой существует.

блаблабла

\section{Некоторые вспомогательные утверждения}

блаблабла

\begin{theorem*}
  $\Q$ счётно.
\end{theorem*}

\begin{proof}
  блаблабла
\end{proof}

\begin{theorem*}
  Если последовательность сходится, то все её подпоследовательности сходятся к тому же числу
\end{theorem*}

\begin{proof}
  блаблабла
\end{proof}

\begin{consequence*}
  Если последовательность имеет две подпоследовательности с различными пределами, она расходится.
\end{consequence*}

\begin{theorem*}
  Пусть $f : \N \rightarrow \R$ и $k \in \N$. Если $f \circ (n \mapsto n + k)$ сходится, то $f$ сходится к тому же числу.
\end{theorem*}

\begin{proof}
  блаблабла
\end{proof}

\begin{theorem*}
  Множество открыто тогда и только тогда, когда замкнуто его дополнение.
\end{theorem*}

\begin{proof}
  блаблабла
\end{proof}

\begin{consequence*}
  Множество замкнуто тогда и только тогда, когда открыто его дополнение.
\end{consequence*}

\begin{proof}
  Действительно, $A = \R \setminus (\R \setminus A)$. Однако, по предыдущей теореме,
  $A = \R \setminus (\R \setminus A)$ замкнуто тогда и только тогда, когда открыто $\R \setminus A$.
\end{proof}

\begin{theorem*}
  Всякое открытое в $\R$ множество представимо как объединение не более чем счётного числа непересекающихся интервалов.
\end{theorem*}

\begin{proof}
  блаблабла
\end{proof}

\begin{theorem*}
  Для всякого $A \subseteq \R$ верно $\overline{A} = \partial A \cup \Int(A) $.
\end{theorem*}

\begin{proof}
  блаблабла
\end{proof}

\begin{lemma*}
  Для всякого $A \subseteq \R$ верно $\partial A = \partial (\R \setminus A)$.
\end{lemma*}

\begin{proof}
Действительно,
\begin{align*}
  & x \in \partial A \\
  \Leftrightarrow\ & \forall \varepsilon > 0: \exists y \in A, y^\prime \in \R \setminus A: |x - y| < \varepsilon \land |x - y^\prime| < \varepsilon \\
  \Leftrightarrow\ & \forall \varepsilon > 0: \exists y^\prime \in \R \setminus A, y \in A: |x - y^\prime| < \varepsilon \land |x - y| < \varepsilon \\
  \Leftrightarrow\ & \forall \varepsilon > 0: \exists y \in \R \setminus A, y^\prime \in \R \setminus (\R \setminus A): |x - y| < \varepsilon \land |x - y^\prime| < \varepsilon \\
  \Leftrightarrow\ & x \in \partial (\R \setminus A)
\end{align*}
\end{proof}

\begin{theorem*}
  Множество всех предельных точек последовательности замкнуто.
\end{theorem*}

\begin{proof}
  блаблабла
\end{proof}

\section{Построение искомой последовательности}

Пусть дано некоторое замкнутое множество $A \subseteq \R$. Тогда множества $\Int(A)$ и $\R \setminus A$ открыты,
потому представимы как объединение не более чем счётного числа непересекающихся интервалов:
\begin{equation*}
  \Int(A) = \bigcup\limits_{i \in I}\ (x_i, y_i),\ \R \setminus A = \bigcup\limits_{j \in J}\ (x^\prime_j, y^\prime_j),\ 
\end{equation*}
где $I \cap J = \varnothing$, $\card(I),\ \card(J) \leq \card(\N)$, $(x_{i_1}, y_{i_1}) \cap (x_{i_2}, y_{i_2}) = \varnothing \Leftrightarrow i_1 \neq i_2$
и, аналогично, $(x^\prime_{j_1}, y^\prime_{j_1}) \cap (x^\prime_{j_2}, y^\prime_{j_2}) = \varnothing \Leftrightarrow j_1 \neq j_2$.

Кроме того, считаем, что $x_i \neq y_i$ и $x^\prime_j \neq y^\prime_j$.

Для каждого индекса $j \in J$ определим последовательность
\begin{equation*}
  b_j = n \mapsto
  \begin{cases}
    x^\prime_j, & n\ \text{чётно}; \\
    y^\prime_j, & n\ \text{нечётно}.
  \end{cases}
\end{equation*}

Далее, поскольку множество рациональных чисел счётно, выполняется
\begin{equation*}
  \card((x_i, y_i)\ \cap\ \Q) \leq \card(\Q) = \card(\N),
\end{equation*}
но, с другой стороны, поскольку интервал $(x_i, y_i)$ непуст, множество $(x_i, y_i)\ \cap\ \Q$ бесконечно.
Следовательно, $\card((x_i, y_i)\ \cap\ \Q) = \card(\N)$, то есть существует биекция $ a_i : \N \rightarrow (x_i, y_i) \cap \Q $ для каждого $i \in I$.

\begin{lemma*}
  Предельными точами последовательности $b_j$ являются только точки $x^\prime_j$ и $y^\prime_j$. 
\end{lemma*}

\begin{proof}
Рассмотрим произвольную подпоследовательность $b_j \circ f$ последовательности $b_j$.
Поскольку $\N$ бесконечно и $b_j(f(\N)) \subseteq \{x^\prime_j, y^\prime_j\}$ (по заданию $b_j$),
$(b_j \circ f)^{-1}(x^\prime_j)$ и $(b_j \circ f)^{-1}(y^\prime_j)$ одновременно не могут быть конечными.

Пусть конечно лишь второе множество. Тогда, отбрасывая достаточное число членов последовательности,
получим постоянную последовательность с пределом $x^\prime_j$. Так как конечное число членов последовательности
не влияют на предел, и исходная последовательность имеет пределом точку $x^\prime_j$.
Аналогичен случай, когда конечно лишь первое множество. Пусть теперь оба множества бесконечны.
В этом случае у $b_j \circ f$ существуют две различные (постоянные) подпоследовательности, потому она предела не имеет.
\end{proof}

\begin{lemma*}
  Множество предельных точек последовательности $a_i$ совпадает с интервалом $(x_i, y_i)$.
\end{lemma*}

\begin{proof}
Пусть $r \in (x_i, y_i)$. Индуктивно построим искомую подпоследовательность.
Возьмём некоторое рациональное число $q_0$ из множества $(x_i, y_i) \setminus \{ r \}$.
Поскольку $a_i$ биективно, числу $q_0$ соответствует некоторое натуральное число $k_0$ такое, что $a_i(k_0) = q_0$.

Рассмотрим проколотую окрестность $(r - q_0; r + q_0) \setminus \{ r \}$. Снова в силу того, что $a_i$ — биекция, в этой окрестности
может содержаться лишь конечное число рациональных точек $q$ таких, что $a_i^{-1}(q) \leq k_0$. Исключая их, выберем в этой окрестности следующую точку $q_1 \in \Q$
с соответствующим номером $k_1 = a_i^{-1}(q_1)$.

Далее рассмотрим окрестность $(r - q_0/2; r + q_0) \setminus \{ r \}$. Исключим из неё все рациональные точки $q$ такие,
что $a_i^{-1}(q) \leq \mathrm{max}\{k_0, k_1\}$ (то есть, в частности, и само $q_1$).

Аналогичным образом выбирая далее точки в окрестностях $(r - q_0/n; r + q_0/n) \setminus \{ r \}$,
получим некоторую строго возрастающую последовательность номеров $k_0 < k_1 < k_2 < ... < k_n < ...$.

Поскольку для $n \geq 2$ выполняется $q_{n-1} \in (r - q_0/n; r + q_0/n)$, то есть $r - q_0/n < q_{n-1} < r + q_0/n$,
и, очевидно, $\underset{n \rightarrow + \infty}{\lim} (r - q_0/n) = \underset{n \rightarrow + \infty}{\lim} (r + q_0/n) = r$,
то, по теореме о двух милиционерах, имеем $\underset{n \rightarrow + \infty}{\lim} q_n = \underset{n \rightarrow + \infty}{\lim} a_i(k_n) = r$, что и требовалось.

\end{proof}

\begin{lemma*}
  $ \partial A = \bigcup\limits_{j \in J} \{ x^\prime_j, y^\prime_j \} $
\end{lemma*}

\begin{proof}
Поскольку интервалы $(x^\prime_j, y^\prime_j)$ попарно не пересекаются, получаем:
\begin{equation*}
  \partial A = \partial (\R \setminus A) = \partial \left( \bigcup\limits_{j \in J}\ (x^\prime_j, y^\prime_j) \right) =
  \bigcup\limits_{j \in J}\ \partial(x^\prime_j, y^\prime_j) = \bigcup\limits_{j \in J} \{ x^\prime_j, y^\prime_j \}
\end{equation*}
\end{proof}

\end{document}
